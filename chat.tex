% -----------------------------------------------------------------------------
% Relatorio_Entrega1.tex – Primeira Entrega Parcial
% Laboratório de Kubernetes & Istio
% -----------------------------------------------------------------------------
\documentclass[12pt,a4paper]{report}

% ---------------------------- Pacotes ----------------------------------------
\usepackage[utf8]{inputenc}
\usepackage[T1]{fontenc}
\usepackage{graphicx}
\usepackage{hyperref}
\usepackage{listings}
\usepackage{minted}
\usepackage{enumitem}
\usepackage{geometry}
\geometry{margin=2.5cm}

% --------------------------- Metadados ---------------------------------------
\title{Relatório – Primeira Entrega Parcial\\Laboratório de Kubernetes & Istio}
\author{Jonas de Araújo Luz Junior\\\small{Instituição: TRE/CE}}
\date{\today}

% ---------------------------- Documento --------------------------------------
\begin{document}

\maketitle
\tableofcontents
\clearpage

% -----------------------------------------------------------------------------
\chapter{Contexto Geral}
Este relatório descreve a execução da \textbf{Primeira Entrega Parcial} do trabalho
prático proposto no Laboratório de Sistemas Distribuídos, cujo objetivo é
provisionar um cluster Kubernetes local, instalar o service‑mesh Istio e
implantar a aplicação de micro‑serviços \emph{Online Boutique}.

% -----------------------------------------------------------------------------
\chapter{Repositório Git}
\begin{itemize}[leftmargin=*]
  \item Repositório oficial do projeto (código, manifestos, evidências):\\
  \url{https://github.com/jonasluz/DIA.kubernetes-istio/tree/main}
\end{itemize}

% -----------------------------------------------------------------------------
\chapter{Ambiente de Trabalho}
\begin{description}[leftmargin=1.5cm]
  \item[Sistema] Fedora 41 (x86\_64) atualizado em \today.
  \item[Recursos] 8 GB RAM, 4 vCPU, 60 GB SSD.
  \item[Container Runtime] Docker \texttt{24.x} (moby‑engine)\footnote{Instalação descrita no Capítulo~\ref{chap:passoapasso}.}
  \item[Cluster] Minikube \texttt{v1.35.0} com Kubernetes \texttt{v1.32.0}.
  \item[Istio] \texttt{1.22.0} (perfil \texttt{demo}).
\end{description}

% -----------------------------------------------------------------------------
\chapter{Passo a Passo da Instalação}
\label{chap:passoapasso}
Os comandos abaixo foram executados sequencialmente em shell \texttt{bash}. Cada
etapa inclui uma breve explicação e um espaço reservado para evidência (log ou
captura de tela).

\section{Atualização do Sistema}
\begin{enumerate}[label=\arabic*.]
  \item \textbf{Atualizar pacotes e utilitários básicos}
\begin{minted}[fontsize=\small]{bash}
sudo dnf upgrade --refresh -y && sudo reboot
after reboot:
sudo dnf install -y curl wget git conntrack jq
author
\end{minted}
\textit{Evidência:} incluir saída de \texttt{dnf upgrade}.\\
\vspace{3cm}
\end{enumerate}

\section{Instalação do Docker}
\begin{enumerate}[label=\arabic*.]
\item\textbf{Adicionar repositório Docker CE e instalar runtime}
\begin{minted}[fontsize=\small]{bash}
sudo dnf install -y dnf-plugins-core
sudo dnf config-manager --add-repo \
  https://download.docker.com/linux/fedora/docker-ce.repo
sudo dnf install -y docker-ce docker-ce-cli containerd.io \
  docker-buildx-plugin docker-compose-plugin
sudo systemctl enable --now docker
sudo usermod -aG docker $(whoami)
\end{minted}
\textit{Evidência:} saída de \texttt{docker --version}.\\
\vspace{3cm}
\end{enumerate}

\section{Instalação do kubectl}
\begin{enumerate}[label=\arabic*.]
\item\textbf{Baixar binário compatível (v1.32.0)}
\begin{minted}[fontsize=\small]{bash}
curl -LO https://dl.k8s.io/release/v1.32.0/bin/linux/amd64/kubectl
sudo install -o root -g root -m 0755 kubectl /usr/local/bin/
rm kubectl
\end{minted}
\textit{Evidência:} saída de \texttt{kubectl version --client}.\\
\vspace{2cm}
\end{enumerate}

\section{Instalação do Minikube}
\begin{enumerate}[label=\arabic*.]
\item\textbf{Baixar Minikube v1.35.0}
\begin{minted}[fontsize=\small]{bash}
curl -LO https://github.com/kubernetes/minikube/releases/download/v1.35.0/minikube-linux-amd64
sudo install minikube-linux-amd64 /usr/local/bin/minikube
rm minikube-linux-amd64
\end{minted}
\item\textbf{Inicializar cluster (driver Docker)}
\begin{minted}[fontsize=\small]{bash}
minikube start --driver=docker --cpus=4 --memory=8192
\end{minted}
\textit{Evidência:} saída de \texttt{minikube status} e \texttt{kubectl get nodes}.\\
\vspace{2cm}
\end{enumerate}

\section{Instalação do Istio}
\begin{enumerate}[label=\arabic*.]
\item\textbf{Download e instalação do Istioctl 1.22.0}
\begin{minted}[fontsize=\small]{bash}
curl -L https://istio.io/downloadIstio | ISTIO_VERSION=1.22.0 sh -
export PATH="$PATH:$HOME/istio-1.22.0/bin"
# opcional: echo no ~/.bashrc
istioctl install --set profile=demo -y
istioctl verify-install
\end{minted}
\textit{Evidência:} lista de pods em \texttt{istio-system}.\\
\vspace{2cm}
\end{enumerate}

\section{Implantação da Online Boutique}
\begin{enumerate}[label=\arabic*.]
\item\textbf{Clonar repositório e implantar namespace \texttt{boutique-base}}
\begin{minted}[fontsize=\small]{bash}
git clone --depth 1 https://github.com/GoogleCloudPlatform/microservices-demo.git
kubectl create namespace boutique-base
kubectl apply -f microservices-demo/release/kubernetes-manifests.yaml -n boutique-base
\end{minted}
\item\textbf{Implantar versão com sidecars Istio (\texttt{boutique-istio})}
\begin{minted}[fontsize=\small]{bash}
kubectl create namespace boutique-istio
kubectl label namespace boutique-istio istio-injection=enabled
kubectl apply -f microservices-demo/release/kubernetes-manifests.yaml -n boutique-istio
\end{minted}
\textit{Evidência:} pods 2/2 containers.\\
\vspace{2cm}
\end{enumerate}

% -----------------------------------------------------------------------------
\chapter{Problemas Encontrados e Soluções}
\begin{itemize}[leftmargin=*]
  \item \textbf{Driver Podman rootless instável}: optou‑se pelo \emph{driver Docker}, que funcionou sem ajustes extras.
  \item \textbf{Aviso de versão do kubectl}: resolvido instalando binário 1.32.
  \item Outros problemas menores foram documentados no repositório.
\end{itemize}

% -----------------------------------------------------------------------------
\chapter{Conclusão}
A primeira entrega foi concluída com êxito: cluster Kubernetes funcional em
Minikube, Istio instalado e aplicação Online Boutique implantada em dois
cenários (com e sem sidecar). As evidências coletadas encontram‑se nos espaços
reservados e em formato completo no diretório \texttt{docs/evidences} do
repositório oficial.

\end{document}