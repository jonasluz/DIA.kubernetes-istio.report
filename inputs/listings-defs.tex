\usepackage{listings}
\usepackage{xcolor}

% Definindo um estilo para shell/Python/etc
\lstdefinestyle{shell}{
  backgroundcolor=\color{gray!10},   % fundo cinza claro
  frame=single,                      % moldura simples
  rulecolor=\color{gray!50},         % cor da moldura
  basicstyle=\ttfamily\small,        % monoespaço tamanho small
  keywordstyle=\color{blue},         % keywords em azul
  commentstyle=\color{green!50!black}, % comentários verde-escuro
  breaklines=true,                   % quebra automática de linhas longas
}

% Definição de estilo para Python
\lstdefinestyle{python}{
  language=Python,                   % destaca sintaxe Python
  backgroundcolor=\color{gray!10},   % fundo cinza claro
  frame=single,                      % moldura ao redor
  rulecolor=\color{gray!50},         % cor da moldura
  basicstyle=\ttfamily\small,        % fonte monoespaço, tamanho small
  keywordstyle=\color{blue},         % palavras-chave em azul
  stringstyle=\color{orange!80!black},% strings em laranja
  commentstyle=\color{green!50!black},% comentários em verde-escuro
  numberstyle=\tiny\color{gray},     % numeração de linhas em cinza
  numbers=left,                      % números à esquerda
  stepnumber=1,                      % numera todas as linhas
  numbersep=5pt,                     % distância dos números ao código
  breaklines=true,                   % quebra linhas longas
  showstringspaces=false             % não marca espaços em strings
}
